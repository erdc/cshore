\documentclass[11pt,oneside]{book}
\usepackage{epsfig}
\usepackage{color}  
\usepackage{oldlfont}
%\usepackage[scanall]{psfrag}
\usepackage{graphics}
\usepackage{amssymb}
\usepackage{amsmath}
\addtolength{\topmargin}{-1cm}
\addtolength{\textheight}{2cm}
\addtolength{\textwidth}{4cm}
\addtolength{\oddsidemargin}{-2cm}
\newcommand{\bc}{\begin{center}}
\newcommand{\ec}{\end{center}}
\newcommand{\bul}{$\bullet$}
\newcommand{\nn}{\nonumber}
\newcommand{\vs}{\vspace*{.5cm}}
\newcommand{\hs}{\mbox{\hspace*{1cm}}}
\newcommand{\be}{\begin{equation}}
\newcommand{\ee}{\end{equation}}
\newcommand{\D}{\Delta}
\newcommand{\ol}{\overline}
\newcommand{\p}{\partial}
\newcommand{\bea}{\begin{eqnarray}}
\newcommand{\eea}{\end{eqnarray}}
\newcommand{\intod}{\int^\eta_{z_b}}
\newcommand{\intoz}{\int^\eta_{z}}
\newcommand{\intozt}{\int^{z_t}_{z_b}}
\newcommand{\pdv}[2]{\frac{\partial \, #1}{\partial #2}}
\newcommand{\bs}[1]{\boldsymbol{#1}}
\newcommand{\bm}[1]{{\mathbf #1}}
\newcommand{\ob}[1]{\overbrace{#1}}
\newcommand{\oa}[1]{\overrightarrow{#1}}
\newcommand{\ph}{\phantom{\;}}
\newcommand{\etal}{ {\it et al.} }
\newcommand{\e}{\epsilon}
\begin{document}
\section*{CSHORE Execution}
CSHORE is distributed as \verb+FORTRAN+ code located in the
\verb+src-repo+ directory.  Two pre-compiled executables for Windows
and Linux, with filenames \verb+cshore_usace_win.out+ and
\verb+CSHORE_USACE_LINUX.out+, respectively, are also included in the
\verb+usace_distribute_bundle/bin/+ directory.  CSHORE is version
reporting is provided at time of execution, with a version date provided as output.

Model execution is completed through either scripting or execution at
the command line.  A Matlab script example is provided in
\verb+usace_distribute_bundle/example_application+ as
\verb+run_model.m+.  Command line execution is accomplished by calling
the CSHORE executable from a directory that includes an input file
named \verb+infile+.  For example, execution from a Windows operating system:
\begin{verbatim}
Z:\wes\cshore-git>cd usace_distribute_bundle\example_application

Z:\wes\cshore-git\usace_distribute_bundle\example_application>..\bin\cshore_usace_win.out
 CSHORE USACE version, 2014 last edit 2024-03-14
 ------------------------------------------------------------
 CSHORE applied to idealized planar slope
 ------------------------------------------------------------
\end{verbatim}

Similarly, execution from a Linux system:

\begin{verbatim}
$ cd usace_distribute_bundle/example_application/
$ ../bin/CSHORE_USACE_LINUX.out 
 CSHORE USACE version, 2014 last edit 2024-03-14                       
 ------------------------------------------------------------          
 CSHORE applied to idealized planar slope                              
 ------------------------------------------------------------          
\end{verbatim}

\section*{CSHORE Parameters}

\noindent The CSHORE family of models has dependencies on the following scalars:

\begin{eqnarray}
d_{50} &=& \mbox{median sediment grain size }\nonumber \\
sg &=& \mbox{sediment specific gravity} \nonumber \\
w_f  &=& \mbox{sediment fall velocity}\nonumber \\
por &=& \mbox{bed porosity}\nonumber \\
e_B &=& \mbox{breaking efficiency} \nonumber \\
e_f &=& \mbox{bottom dissipation efficiency}\nonumber \\
slp &=& \mbox{suspended load parameter}\nonumber \\  
slpot &=& \mbox{suspended load parameter for over-topping}\nonumber \\  
blp &=& \mbox{wave-related bed load parameter}\nonumber \\  
f_w &=& \mbox{wave friction factor} \nonumber\\
\gamma &=& \mbox{ratio of breaking wave height to water depth} \nonumber
\end{eqnarray}
\noindent {\bf Physical Parameters}\\ Some parameters are physical and
are prescribed by field measurements or estimation.  The median grain
size, $d_{50}$, for instance, is specified according to measured or
assumed data.  Likewise the sediment fall velocity $w_f$ and porosity
$por$ are physical attributes of the model domain.  Typical values of
these parameters is given:
\begin{table}[h]
\begin{center}
%\begin{tabular}{||p{5cm}|c|p{5cm}||}\hline
\begin{tabular}{||c|c|c||}\hline
Parameter& Typical Value   & Units  \\ \hline \hline
$d_{50}$ & 0.20 & mm \\ \hline
$w_f$ & 0.026 & m/s \\ \hline
$sg$ & 2.65 &  \\ \hline
$por$ & 0.40 &  \\ \hline
\end{tabular}
\end{center}
\caption{Physical parameters}
\end{table}
\clearpage
\noindent {\bf Empirical and Numerical Parameters}\\ The 1-D CSHORE model
formulation includes empirical devices for estimation of nearshore
hydrodynamics and sediment transport.  Typical values for the
parameters is provided along with a range of acceptable values.  It
should be noted that the breaking model in CSHORE relies on a
user-supplied $\gamma$, the ratio of wave height to water depth in the
saturated breaking region.  Strictly speaking, this is not an
empirical model parameter, and data from the surf zone is usually
available in the laboratory for guidance.  In the typical application
with field conditions, however, it is not practical to collect surf
zone information, and typical values are provided in Table \ref{fig1}.

\begin{table}[h]
\begin{center}
\begin{tabular}{||c|c|c||}\hline
Parameter& Typical Value   & Range  \\ \hline \hline
$dx$ & 1 & 0.1 (lab) -- 1(field) \\ \hline
$\gamma$ & 0.7 & 0.5 -- 0.9 \\ \hline
$e_{B}$ & 0.005 & 0.001 -- 0.01 \\ \hline
$e_{f}$ & 0.01 &  \\ \hline
$slp$ & 0.5 & 0.2 -- 0.5 \\ \hline
$slpot$ & 0.1 & 0.05 -- 0.2 \\ \hline
$\tan \phi$ & 0.63 & 0.63 \\ \hline
$blp$ & 0.001 & 0.0005 -- 0.002 \\ \hline
$f_w$ & 0.015 & 0.005 -- 0.03 \\ \hline
$rwh$ & 0.03 & 0.01 (lab) -- .05(field) \\ \hline
\end{tabular}
\end{center}
\caption{Empirical  parameters}
\label{fig1}
\end{table}
\clearpage
\noindent {\bf Input file structure}\\ The CSHORE model developers,
over time, have added capabilities and code branches to extend the
range of application.  The optional processes are included with an
array of logical parameters that are given below.  It is suggested, at
present, to provide the user with a limited set of options to avoid
unadvised application before complete scrutiny by the USACE.  The
following four tables detail the input file structure required by
CSHORE.  

\begin{table}[h]
\begin{center}
%\begin{tabular}{||p{5cm}|c|p{5cm}||}\hline
\begin{tabular}{||c|c|p{8cm}||}\hline
Parameter& Allowed Values   & Meaning and Conditional  \\ \hline \hline
ILINE & 1 & Single transect  \\ \hline
IPROFL & 0, 1 &  0=no sediment transport, 1=sediment transport\\ \hline
ISEDAV & 0 & unlimited sediment availability, Conditional on IPROFL = 1 \\ \hline
IPERM & 0 & neglect permeability \\ \hline
IOVER & 1 & Allow overtopping and compute runup statistics \\ \hline
IWTRAN & 0 & No standing water in landward zone, Conditional on IOVER = 1 \\ \hline
IPOND & 0 & No ridge and runnel, Conditional on IOVER = 1 \\ \hline
INFILT & 0 & No inflitration landward of dune crest, Conditional on IOVER = 1 and IWTRAN=0 \\ \hline
IWCINT & 0 & No wave-current interaction\\ \hline
IROLL & 0 & No roller effect\\ \hline
IWIND & 0 & No wind effect\\ \hline
ITIDE & 0 & No pressure effect\\ \hline
IVEG & 0 & No vegitation effect\\ \hline
DXC & $dx$ & see above\\ \hline
GAMMA & $\gamma$ & see above\\ \hline
D50 \; WF \; SG & $d_{50} \; w_f \; sg$ & see above, Conditional on IPROFL = 1\\ \hline
EFFB \; EFFF \; SLP \; SLPOT & $ e_{b} \; e_{f} \; slp \; slpot$ & see above, Conditional on IPROFL = 1\\ \hline
TANPHI  BLP & $ \tan \phi  \; blp $ & see above, Conditional on IPROFL = 1\\ \hline
RWH & $rwh$ & See above\\ \hline
ILAB & 0 & Assume continuous and bounded boundary condition data\\ \hline
NWAVE &  & Number of wave conditions. Provide $nwave+1$ conditions for interpolation with ILAB=0\\ \hline
NSURGE &  & Number of water level conditions. Provide $nsurge+1$ conditions for interpolation with ILAB=0\\ \hline
\end{tabular}
\end{center}
%% \end{table}
%% \begin{table}[h]
\begin{center}
\begin{tabular}{||c|c|c|c||}\hline
time(s) & wave period(s) & root-mean-square wave height(m)& wave setup(m)\\ \hline
\end{tabular}
\end{center}
%% \end{table}
%% \begin{table}[h]
\begin{center}
\begin{tabular}{||c|c||}\hline
time(s) & water level(m)\\ \hline
\end{tabular}
\end{center}
%% \end{table}
%% \begin{table}[h]
\begin{center}
%\begin{tabular}{||p{5cm}|c|p{5cm}||}\hline
\begin{tabular}{||c|c|p{8cm}||}\hline
Parameter& Allowed Values   & Meaning and Conditional  \\ \hline \hline
NPINP &  & Number of bottom position data points  \\ \hline
\end{tabular}
\end{center}
%% \end{table}
%% \begin{table}[h]
\begin{center}
\begin{tabular}{||c|c|c||}\hline
$x$(m) & $z_b$(m) & $f_w$\\ \hline
\end{tabular}
\end{center}
\end{table}

\clearpage

\noindent {\bf Sample INFILE}
For cases of fixed bed, IPROFL=0
\begin{verbatim}
3 
------------------------------------------------------------ 
CSHORE applied to idealized planar slope 
------------------------------------------------------------ 
1                                         ->ILINE
0                                         ->IPROFL
0                                         ->IPERM
1                                         ->IOVER
0                                         ->IWTRAN
0                                         ->IPOND
0                                         ->IWCINT
0                                         ->IROLL 
0                                         ->IWIND 
0                                         ->ITIDE 
0                                         ->IVEG 
     1.0000                                ->DXC
     0.8000                                ->GAMMA 
     0.0200                               ->RWH 
0                                         ->ILAB
5                                         ->NWAVE 
5                                         ->NSURGE 
       0.00     8.0000     2.1000     0.0000
    3600.00     8.0000     2.2000     0.0000
    7200.00     8.0000     2.3000     0.0000
   10800.00     8.0000     2.4000     0.0000
   14400.00     8.0000     2.5000     0.0000
   18000.00     8.0000     2.6000     0.0000
       0.00     0.0000
    3600.00     0.5000
    7200.00     0.8660
   10800.00     1.0000
   14400.00     0.8660
   18000.00     0.5000
301                                  ->NBINP 
     0.0000    -8.0000     0.0150
     1.0000    -8.0000     0.0150
     2.0000    -8.0000     0.0150
     3.0000    -8.0000     0.0150
     4.0000    -8.0000     0.0150
     5.0000    -8.0000     0.0150
     .
     .
     .
   297.0000     7.7600     0.0150
   298.0000     7.8400     0.0150
   299.0000     7.9200     0.0150
   300.0000     8.0000     0.0150

\end{verbatim}

For cases including sediment transport, IPROFL=1
\begin{verbatim}
3 
------------------------------------------------------------ 
CSHORE applied to idealized planar slope 
------------------------------------------------------------ 
1                                         ->ILINE
1                                         ->IPROFL
0                                         ->ISEDAV
0                                         ->IPERM
1                                         ->IOVER
0                                         ->IWTRAN
0                                         ->IPOND
0                                         ->INFILT
0                                         ->IWCINT
0                                         ->IROLL 
0                                         ->IWIND 
0                                         ->ITIDE 
0                                         ->IVEG 
     1.0000                                ->DXC
     0.8000                                ->GAMMA 
     0.3000     0.0448     2.6500         ->D50 WF SG
     0.0050     0.0100     0.5000     0.1000       ->EFFB EFFF SLP  SLPOT
     0.6300     0.0010                    ->TANPHI BLP
     0.0200                               ->RWH 
0                                         ->ILAB
5                                         ->NWAVE 
5                                         ->NSURGE 
       0.00     8.0000     2.1000     0.0000
    3600.00     8.0000     2.2000     0.0000
    7200.00     8.0000     2.3000     0.0000
   10800.00     8.0000     2.4000     0.0000
   14400.00     8.0000     2.5000     0.0000
   18000.00     8.0000     2.6000     0.0000
       0.00     0.0000
    3600.00     0.5000
    7200.00     0.8660
   10800.00     1.0000
   14400.00     0.8660
   18000.00     0.5000
301                                  ->NBINP 
     0.0000    -8.0000     0.0150
     1.0000    -8.0000     0.0150
     2.0000    -8.0000     0.0150
     3.0000    -8.0000     0.0150
     4.0000    -8.0000     0.0150
     5.0000    -8.0000     0.0150
     .
     .
     .
   297.0000     7.7600     0.0150
   298.0000     7.8400     0.0150
   299.0000     7.9200     0.0150
   300.0000     8.0000     0.0150

\end{verbatim}


\end{document}
